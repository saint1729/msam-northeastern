\documentclass[letterpaper]{article}
\pagestyle{empty}
\textheight=8in
\textwidth=6in
\hoffset=-.5in
\newcommand{\R}{{\mathbb R}}
%\input C:/W-proc/newcomm

\usepackage{enumerate}
\usepackage{amssymb}
\usepackage{amsmath}
\begin{document}
 \centerline{McGill University}
 \centerline{Math 325A: Differential Equations}

 \bk
 \centerline{\bf \large{ LECTURE 1: INTRODUCTION} }
 \vskip .1in
 \centerline{(Text: Sections 1.1, 1.2)} \vskip .1in

\section{Definitions and Basic Concepts}

\subsection{Ordinary Differential Equation (ODE)} An equation
involving the derivatives of an unknown function $y$ of a single
variable $x$ over an interval $ x\in (I)$.

\subsection{Solution} Any function $y=f(x)$ which
satisfies this equation over the interval $(I)$ is called a
solution of the ODE.

For example, $y=e^{2x}$ is a solution of the ODE
$$y'=2y$$

and $y=\sin(x^2)$ is a solution of the ODE
$$xy''-y'+4x^3y=0.$$

\subsection{Order  $n$ of the DE} An ODE is said to be order $n$,
 if $y^{(n)}$ is the highest order
derivative occurring in the equation. The simplest first order ODE
is $y'=g(x)$.

 The most general form of an $n$-th order ODE is
$$F(x,y,y',\ldots,y^{(n)})=0$$
with $F$ a function of $n+2$ variables $x,u_0,u_1,\ldots,u_n$. The
equations
$$ xy''+y=x^3,\quad y'+y^2=0,\quad y'''+2y'+y=0$$
are examples of ODE's of second order, first order and third order
respectively with respectively
$$F(x,u_0,u_1,u_2)=xu_2+u_0-x^3 ,\ \
 F(x,u_0,u_1)=u_1+u_0^2,\ \ F(x,u_0,u_1,u_2,u_3)=u_3+2u_1+u_0.$$

\subsection{Linear Equation:} If the function $F$ is linear in the
variables $u_0,u_1,\ldots,u_n$ the ODE is
 said to be {\bf linear}. If, in addition, $F$ is homogeneous then the ODE is said
to be homogeneous. The first of the above examples above is linear
are linear, the second is non-linear and the third is linear and
homogeneous. The general $n$-th order linear ODE can be written
$$
a_n(x)\frac{d^ny}{dx^n}+a_{n-1}(x)\frac{d^{n-1}y}{dx^{n-1}}+\cdots+
a_1(x)\frac{dy}{dx}+a_0(x)y=b(x).$$

\subsection{Homogeneous Linear Equation:}
 The linear DE is homogeneous, if and only if $b(x)\equiv 0$. Linear
homogeneous equations have the important property that linear
combinations of solutions are also solutions. In other words, if
$y_1,y_2,\ldots,y_m$ are solutions and $c_1,c_2,\ldots,c_m$ are
constants then
$$c_1y_1+c_2y_2+\cdots+c_my_m$$
is also a solution.

\subsection{Partial Differential Equation (PDE)} An
equation involving the partial derivatives of a function of more
than one variable is called PED. The concepts of linearity and
homogeneity can be extended to PDE's. The general second order
linear PDE in two variables $x,y$ is
$$
a(x,y)\frac{\partial^2u}{\partial
x^2}+b(x,y)\frac{\partial^2u}{\partial x\partial
y}+c(x,y)\frac{\partial^2u}{\partial y^2}+d(x,y)\frac{\partial
u}{\partial x}+e(x,y)\frac{\partial u}{\partial y}+f(x,y)u=g(x,y).
$$
Laplace's equation
$$\frac{\partial^2 u}{{\partial x}^2}+
\frac{\partial^2 u}{{\partial y}^2}=0$$ is a linear, homogeneous
PDE of order $2$. The functions $u=\log(x^2+y^2)$, $u=xy$,
$u=x^2-y^2$ are examples of solutions of Laplace's equation. We
will not study PDE's systematically in this course.

\subsection{General Solution of a Linear Differential Equation} It
represents the set of all solutions, i.e., the set of all
functions which satisfy the equation in the interval (I).
 For
example, the general solution of the differential equation
$y'=3x^2$ is $y=x^3+C$ where $C$ is an arbitrary constant. The
constant $C$ is the value of $y$ at $x=0$. This {\bf initial
condition} completely determines the solution. More generally, one
easily shows that given $a,b$ there is a unique solution $y$ of
the differential equation with $y(a)=b$. Geometrically, this means
that the one-parameter family of curves $y=x^2 + C$ do not
intersect one another and they fill up the plane $\R^2$.

\subsection{A System of ODE's}
$$
\barray
 y_1'&=G_1(x,y_1,y_2,\ldots,y_n)\\
 y_2'&=G_2(x,y_1,y_2,\ldots,y_n)\\
 \vdots\\
  y_n'&=G_n(x,y_1,y_2,\ldots,y_n)
\earray
$$
  An $n$-th order ODE of the form
$y^{(n)}=G(x,y,y',\ldots,y^{n-1})$ can be transformed in the form
of the system of first order DE's. If we introduce dependant
variables $y_1=y,y_2=y',\ldots ,y_n=y^{n-1}$ we obtain the
equivalent system of first order equations
\begin{align*}
y_1'&=y_2,\\ y_2'&=y_3,\\ \vdots& \\y_n'&=G(x,y_1,y_2,\ldots,y_n).
\end{align*}
For example, the ODE $y''=y$ is equivalent to the system
\begin{align*}
y_1'&=y_2,\\ y_2'&=y_1.
\end{align*}

 In this way the study of $n$-th order equations can be
reduced to the study of systems of first order equations. Some
times, one called the latter as the {\bf normal form} of the
$n$-th order ODE.
 Systems of equations arise in the
study of the motion of particles. For example, if $P(x,y)$ is the
position of a particle of mass $m$ at time $t$, moving in a plane
under the action of the force field $(f(x,y),g(x,y)$, we have
\begin{align*}
m\frac{d^2x}{dt^2}&= f(x,y),\\ m\frac{d^2y}{dt^2}&=g(x,y).
\end{align*}
This is a second order system.

The general first order ODE in normal form is
$$
y'=F(x,y).
$$
If $F$ and $\frac{\partial F}{\partial y}$ are continuous one can
show that, given $a,b$, there is a unique solution with $y(a)=b$.
Describing this solution is not an easy task and there are a
variety of ways to do this. The dependence of the solution on
initial conditions is also an important question as the initial
values may be only known approximately.

The non-linear ODE $yy'=4x$ is not in normal form but can be
brought to normal form
$$
y'=\frac{4x}{y}.
$$
by dividing both sides by $y$.


\section{The Approaches of Finding Solutions of ODE}

 \subsection{Analytical Approaches}
 \bitem
 \item
 Analytical solution methods: finding the exact form of solutions;
 \item
 Geometrical methods: finding the qualitative behavior of
 solutions;
 \item
 Asymptotic methods: finding the asymptotic form of the solution,
 which gives good approximation of the exact solution.
\eitem

 \subsection{ Numerical Approaches}
 \bitem
\item
Numerical algorithms --- numerical methods;
\item
Symbolic manipulators --- Maple, MATHEMATICA, MacSyma.

 \eitem


This course mainly discuss the analytical approaches and mainly on
analytical solution methods.

\end{document}
