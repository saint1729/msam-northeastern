\documentclass[11pt]{amsart}
\usepackage[margin=1in]{geometry}
\usepackage[all]{xy}


\usepackage{amsmath,amsthm,amssymb,color,latexsym}
\usepackage{geometry}        
\geometry{letterpaper}    
\usepackage{graphicx}

\usepackage{listings}
\usepackage{xcolor}

\usepackage[export]{adjustbox}

\definecolor{codegreen}{rgb}{0,0.6,0}
\definecolor{codegray}{rgb}{0.5,0.5,0.5}
\definecolor{codepurple}{rgb}{0.58,0,0.82}
\definecolor{backcolour}{rgb}{0.95,0.95,0.92}

\lstdefinestyle{mystyle}{
  backgroundcolor=\color{backcolour},   commentstyle=\color{codegreen},
  keywordstyle=\color{magenta},
  numberstyle=\tiny\color{codegray},
  stringstyle=\color{codepurple},
  basicstyle=\ttfamily\footnotesize,
  breakatwhitespace=false,         
  breaklines=true,                 
  captionpos=b,                    
  keepspaces=true,                 
  numbers=left,                    
  numbersep=5pt,                  
  showspaces=false,                
  showstringspaces=false,
  showtabs=false,                  
  tabsize=2
}

\lstset{style=mystyle}

\usepackage{graphicx}
\graphicspath{ {./images/} }


\newtheorem{problem}{Problem}

\newenvironment{solution}[1][\it{Solution}]{\textbf{#1. } }{$\square$}














\usepackage{txfonts}
\usepackage{amsmath,amsthm,verbatim}
\usepackage{amscd,amssymb}
\usepackage[all]{xy} 
\usepackage{graphicx,calrsfs} 
\usepackage[colorlinks,plainpages,urlcolor=blue]{hyperref} 
\usepackage{breakurl}
\def\UrlBreaks{\do\/\do-}\usepackage{bbm}
\usepackage{bbm}
\usepackage{mathrsfs}
\usepackage{xstring}
\usepackage{tikz}
\usetikzlibrary{matrix,arrows,decorations.pathmorphing,calc,shapes}
\usepackage{mathtools}
\usepackage{tcolorbox}
\usepackage{wrapfig} 

\setlength{\parskip}{0.1 in}
\setlength{\parindent}{0pt}

%\usepackage[utf8]{inputenc}

\usepackage{listings}
\usepackage{xcolor}

\definecolor{codegreen}{rgb}{0,0.6,0}
\definecolor{codegray}{rgb}{0.5,0.5,0.5}
\definecolor{codepurple}{rgb}{0.58,0,0.82}
\definecolor{backcolour}{rgb}{0.95,0.95,0.92}

\lstdefinestyle{mystyle}{
   % backgroundcolor=\color{backcolour},   
    commentstyle=\color{codegreen},
    keywordstyle=\color{magenta},
    numberstyle=\tiny\color{codegray},
    stringstyle=\color{codepurple},
    basicstyle=\ttfamily\footnotesize,
    breakatwhitespace=false,         
    breaklines=true,                 
    captionpos=b,                    
    keepspaces=true,                 
    numbers=left,                    
    numbersep=5pt,                  
    showspaces=false,                
    showstringspaces=false,
    showtabs=false,                  
    tabsize=2
}

 \lstset{style=mystyle}



\topmargin=0in
\textwidth6.6in
\textheight8.6in
\oddsidemargin=-0.1in
\evensidemargin=-0.1in

\newtheorem{theorem}{Theorem} 
\newtheorem{corollary}[theorem]{Corollary}
\newtheorem{lemma}[theorem]{Lemma}
\newtheorem{prop}[theorem]{Proposition}

\theoremstyle{definition}
\newtheorem{definition}[theorem]{Definition}
%\newtheorem{example}[theorem]{Example}
\newtheorem{remark}[theorem]{Remark}
\newtheorem{conjecture}[theorem]{Conjecture}
\newtheorem{question}[theorem]{Question}
\newtheorem{ex}[theorem]{Example}

 
\newcommand{\N}{\mathbb{N}}
\newcommand{\R}{\mathbb{R}}
\newcommand{\Q}{\mathbb{Q}}
\newcommand{\C}{\mathbb{C}}
\newcommand{\Z}{\mathbb{Z}}
\newcommand{\F}{\mathbb{F}}
\newcommand{\kk}{\mathbbm{k}}
\newcommand{\mm}{\mathbbm{m}}

\newcommand{\vtheta}{\vec{\theta}}


\newcommand{\va}{\vec{a}}
\newcommand{\vb}{\vec{b}}
\newcommand{\vc}{\vec{c}}
\newcommand{\vd}{\vec{d}}
\newcommand{\vh}{\vec{h}}
\newcommand{\e}{\vec{e}}
\newcommand{\vu}{\vec{u}}
\newcommand{\vv}{\vec{v}}
\newcommand{\vw}{\vec{w}}
\newcommand{\vx}{\vec{x}}
\newcommand{\vy}{\vec{y}}
\newcommand{\vz}{\vec{z}}
\newcommand{\Span}{\mathrm{Span}}
\newcommand{\red}{\textcolor{red}}
\newcommand{\blue}{\textcolor{blue}}
\newcommand{\green}{\textcolor{green}}
\newcommand{\ifif}{\textcolor{blue}{ if and only if }} 


\newcommand{\cB}{\mathcal{B}}
\newcommand{\cE}{\mathcal{E}}
\newcommand{\cC}{\mathcal{C}}
\newcommand{\cD}{\mathcal{D}}

\DeclareMathOperator{\im}{im}
\DeclareMathOperator{\codim}{codim}
\DeclareMathOperator{\coker}{coker}
 \DeclareMathOperator{\rank}{rank}
\DeclareMathOperator{\Nul}{\mathrm{Nul}}
\DeclareMathOperator{\Co}{\mathrm{Col}}
\DeclareMathOperator{\Ro}{\mathrm{Row}}
\DeclareMathOperator{\Ker}{\mathrm{Ker}}
\DeclareMathOperator{\Ima}{\mathrm{Im}}
\DeclareMathOperator{\dist}{\mathrm{dist}}
\DeclareMathOperator{\proj}{\mathrm{proj}}
\DeclareMathOperator{\refl}{\mathrm{ref}}
\DeclareMathOperator{\sign}{\mathrm{sign}}

\DeclareMathOperator{\id}{id}
\DeclareMathOperator{\tr}{tr}

\newcommand{\ef}{\textbf{ref}}
\newcommand{\rref}{\textbf{rref}}

\newcommand{\pdf}{\textbf{pdf }}
\newcommand{\pmf}{\textbf{pmf }}
\newcommand{\cdf}{\textbf{cdf }}

\newcommand{\normalcdf}{\textbf{normalcdf}}
\newcommand{\invNorm}{\textbf{invNorm}}
\newcommand{\invT}{\textbf{invT}}
\newcommand{\popdf}{\textbf{poissonpdf}}
\newcommand{\pocdf}{\textbf{poissoncdf}}

\DeclareMathOperator{\Var}{\mathrm{Var}}
\DeclareMathOperator{\Cov}{\mathrm{Cov}}
\DeclareMathOperator{\Corr}{\mathrm{Corr}}

\DeclareMathOperator{\Normal}{\mathrm{Normal}}

 \pgfmathdeclarefunction{gauss}{2}{%
  \pgfmathparse{1/(#2*sqrt(2*pi))*exp(-((x-#1)^2)/(2*#2^2))}%
}


  
 









\begin{document}
\noindent MATH 7243, Spring 2022\hfill Homework 1.  Matrix calculus\\
Sakshi Suman (01/29/2022)\hfill suman.sak@northeastern.edu

\hrulefill


\begin{problem}
Assume $\vx\in \R^n$,  $A\in \R^{m\times n}$ and $\vb\in \R^m$. Let $f(\vx)=\vb^T A \vx$. Find  $\nabla f$. 
\end{problem}

\begin{solution}
Given,  $\vx\in \R^n$,  $A\in \R^{m\times n}$, $\vb\in \R^m$, $f(\vx)=\vb^T A \vx$.
Let,  $$\vb = \begin{bmatrix}b_1\\\vdots\\b_m\end{bmatrix} \text{and } A = \left[\begin{array}{ccc}a_{11} & \cdots & a_{1n} \\ \vdots & \ddots & \vdots \\ a_{m1} & \cdots & a_{mn}\end{array}\right]$$
$$
\vb^TA = \begin{bmatrix}\sum_{i = 1}^{m}b_i \cdot a_{i1} \cdots \sum_{i = 1}^{m}b_i \cdot a_{in}\end{bmatrix}
$$
Therefore,
$$
f(\vx) = \vb^TA\vx = \sum_{j=1}^{n}\left(\left(\sum_{i=1}^{m}b_i \cdot a_{ij}\right)x_j\right)
$$
\begin{align*}
\nabla f &= \begin{bmatrix}\dfrac{\partial f}{\partial x_1}\\\vdots\\\dfrac{\partial f}{\partial x_n}\end{bmatrix}\\
&= \begin{bmatrix}\sum_{i = 1}^{m}b_i \cdot a_{i1}\\\vdots\\\sum_{i = 1}^{m}b_i \cdot a_{in}\end{bmatrix}\\
&= (\vb^TA)^T\\
&= \boxed{A^T\vb}
\end{align*}

\end{solution} 





\begin{problem}
Assume $\vx\in \R^n$. Find $\dfrac{\partial \vx^T\vx}{\partial \vx}$.
\end{problem}
\begin{solution}
Let, $\vx = \begin{bmatrix}x_1\\\vdots\\x_n\end{bmatrix} \implies  \vx^T\vx = \sum_{i=1}^{n}x_i^2$.

Therefore,
\begin{align*}\dfrac{\partial{\vx^T\vx}}{\partial\vx} &= \begin{bmatrix}2\cdot x_1\\\vdots\\ 2 \cdot x_n\end{bmatrix}\\
&= \boxed{2 \cdot \vx}
\end{align*}
\end{solution}


\begin{problem}
Assume $\vx$ and $\va\in \R^n$. Find $\dfrac{\partial \big(\vx^T\va\big)^2}{\partial \vx}$
\end{problem}

\begin{solution}
Let, $\vx = \begin{bmatrix}x_1\\\vdots\\x_n\end{bmatrix}\text{, } \va = \begin{bmatrix}a_1\\\vdots\\a_n\end{bmatrix} \implies \vx^T\va = \sum_{i=1}^{n}x_i \cdot a_i \implies (\vx^T\va)^2 = (\sum_{i=1}^{n}x_i \cdot a_i)^2$

Therefore,

\begin{align*}
\dfrac{\partial \big(\vx^T\va\big)^2}{\partial \vx} &= \begin{bmatrix}\dfrac{\partial \big(\vx^T\va\big)^2}{\partial x_1}\\\vdots\\\dfrac{\partial \big(\vx^T\va\big)^2}{\partial x_n}\end{bmatrix}\\
&= \begin{bmatrix}2 \cdot (\sum_{i=1}^{n}x_i \cdot a_i) \cdot a_1\\\vdots\\ 2 \cdot (\sum_{i=1}^{n}x_i \cdot a_i) \cdot a_n\end{bmatrix}\\
&= \boxed{2 \cdot (\vx^T \va) \cdot \va}
\end{align*}



\end{solution}



\begin{problem}
Suppose $\vx: \R^n\to \R^m$ is a map  sending $\vz\in \R^n$ to  $\vx(\vz)\in \R^m$. Similarly, 
suppose $\vy: \R^n\to \R^m$ and $A$ is an $m\times m$ constant matrix.
Prove that $\dfrac{\partial \big(\vy^TA\vx\big)}{\partial \vz}=
\dfrac{\partial \vy}{\partial \vz} A \vx+ \dfrac{\partial \vx}{\partial \vz} A^T \vy
$ 
\end{problem}

\begin{solution}
Let, $\vz = \begin{bmatrix}z_1\\\vdots\\z_n\end{bmatrix} \text{, } \vx(\vz) = \begin{bmatrix}x_1\\\vdots\\x_m\end{bmatrix} \text{, } \vy(\vz) = \begin{bmatrix}y_1\\\vdots\\y_m\end{bmatrix} \text{, } A = \left[\begin{array}{ccc}a_{11} & \cdots & a_{1m} \\ \vdots & \ddots & \vdots \\ a_{m1} & \cdots & a_{mm}\end{array}\right]$

Also let, $G = A^T\vy, F = \vx \text{ and } H = G^TF = \vy^TA\vx$. As per lecture notes,\\
\begin{align}
\dfrac{\partial H}{\partial \vz} = \dfrac{\partial G}{\partial \vz} F + \dfrac{\partial F}{\partial \vz} G
\end{align}

Therefore,
\begin{align}
\dfrac{\partial F}{\partial \vz} = \dfrac{\partial \vx}{\partial \vz}
\end{align}

Consider,
\begin{align*}
\dfrac{\partial G}{\partial \vz} &= \dfrac{\partial (A^T\vy)}{\partial \vz}\\
&= \dfrac{\partial}{\partial \vz} \left( \begin{bmatrix}\sum_{i=1}^{m}a_{i1} \cdot y_i\\\vdots\\\sum_{i=1}^{m}a_{im} \cdot y_i\end{bmatrix} \right)\\
&= \left[\begin{array}{ccc}\dfrac{\partial}{\partial z_1} \left( \sum_{i=1}^{m}a_{i1} \cdot y_i \right) & \cdots & \dfrac{\partial}{\partial z_1} \left( \sum_{i=1}^{m}a_{im} \cdot y_i \right) \\ \vdots & \ddots & \vdots \\ \dfrac{\partial}{\partial z_n} \left( \sum_{i=1}^{m}a_{i1} \cdot y_i \right) & \cdots & \dfrac{\partial}{\partial z_n} \left( \sum_{i=1}^{m}a_{im} \cdot y_i \right)\end{array}\right]\\
&= \left[\begin{array}{ccc} \sum_{i=1}^{m}\left(a_{i1} \cdot \dfrac{\partial y_i}{\partial z_1} \right) & \cdots & \sum_{i=1}^{m} \left( a_{im}  \cdot \dfrac{\partial y_i}{\partial z_1} \right) \\ \vdots & \ddots & \vdots \\ \sum_{i=1}^{m}\left(a_{i1} \cdot \dfrac{\partial y_i}{\partial z_n} \right) & \cdots & \sum_{i=1}^{m}\left(a_{im} \cdot \dfrac{\partial y_i}{\partial z_n} \right)\end{array}\right]\\
&= \left[\begin{array}{ccc}\dfrac{\partial y_1}{\partial z_1} & \cdots & \dfrac{\partial y_m}{\partial z_1} \\ \vdots & \ddots & \vdots \\ \dfrac{\partial y_1}{\partial z_n} & \cdots & \dfrac{\partial y_m}{\partial z_n}\end{array}\right] \left[\begin{array}{ccc}a_{11} & \cdots & a_{1m} \\ \vdots & \ddots & \vdots \\ a_{m1} & \cdots & a_{mm}\end{array}\right]\\
&= \frac{\partial \vy}{\partial \vz}A
\end{align*}

Therefore,

\begin{align}
\dfrac{\partial G}{\partial \vz} = \dfrac{\partial \vy}{\partial \vz} A
\end{align}

From equations (1), (2) and (3) above,

\begin{align*}
\boxed{\dfrac{\partial (\vy^TA\vx)}{\partial \vz} = \dfrac{\partial \vy}{\partial \vz} A \vx + \dfrac{\partial \vx}{\partial \vz} A^T \vy}
\end{align*}

\end{solution}


\begin{problem}
Suppose $A(x): \R\to \R^{n\times n}$ is a map from $\R$ to $\R^{n
\times n}.$ 

Show that if $A(x)$ is invertible, then $
\dfrac{dA^{-1}}{dx}=-A^{-1} \dfrac{dA}{dx} A^{-1}
$
\end{problem}
\begin{solution}
Given $A(x): \R\to \R^{n\times n}$. We have,  $AA^{-1} = I_n$. Differentiating both sides w.r.t $x$ gives,

\begin{align*}
&\frac{d }{dx}\left(AA^{-1}\right) = \frac{d I_n}{dx}\\
\implies &\dfrac{dA}{dx} A^{-1} + A \dfrac{dA^{-1}}{dx} = 0\\
\implies &\boxed{\dfrac{dA^{-1}}{dx}=-A^{-1} \dfrac{dA}{dx} A^{-1}}
\end{align*}



\end{solution}



\begin{problem}
 Let $\vx$ and $\beta\in \R^p$. Prove that 
 $
\dfrac{\partial \vx^T\beta}{\partial \vx}=\beta 
  $
\end{problem}
\begin{solution}
\begin{align*}
\dfrac{\partial \vx^T\beta}{\partial \vx} &= \dfrac{\partial \left(\sum_{i=1}^{p} x_i\beta_i \right)}{\partial \vx}\\
&= \begin{bmatrix}\dfrac{\partial \left( \sum_{i=1}^{p}x_i \cdot \beta_i \right)}{\partial x_1}\\\vdots\\\dfrac{\partial \left( \sum_{i=1}^{p}x_i \cdot \beta_i \right)}{\partial x_p}\end{bmatrix}\\
&= \begin{bmatrix}\beta_1\\\vdots\\ \beta_p\end{bmatrix}\\
&= \boxed{\beta}
\end{align*}

\end{solution}



\begin{problem}
 Assume that $Y$ is an $n$ vector but assume that $Y$ depends on $X$ and $X$ depends on
some $Z \in \R^q$. Show that
 $$
 \dfrac{\partial  Y}{\partial Z}= 
  \dfrac{\partial  X}{\partial Z}
   \dfrac{\partial  Y}{\partial X}
 $$
 Does the order matter? 
 
 Hint: This means that $X: \R^q\to \R^p$ and $Y: \R^p\to \R^n$.
 
\end{problem}
\begin{solution}
Given, $Y(X) \in \R^n$ depends on $X(Z) \in \R^p$ for some $Z \in \R^q$.

\begin{align*}
\dfrac{\partial Y}{\partial Z} &= \left[\begin{array}{ccc}\dfrac{\partial \left(y_1(x_1, \cdots x_p)\right)}{\partial z_1} & \cdots & \dfrac{\partial \left(y_m(x_1, \cdots x_p)\right)}{\partial z_1} \\ \vdots & \ddots & \vdots \\ \dfrac{\partial \left(y_1(x_1, \cdots x_p)\right)}{\partial z_n} & \cdots & \dfrac{\partial \left(y_m(x_1, \cdots x_p)\right)}{\partial z_n}\end{array}\right]\\
&= \left[\begin{array}{ccc}\sum_{i=1}^{p}\dfrac{\partial x_i}{\partial z_1} \cdot \dfrac{\partial y_1}{\partial x_i} & \cdots & \sum_{i=1}^{p} \dfrac{\partial x_i}{\partial z_1} \cdot \dfrac{\partial y_m}{\partial x_i} \\ \vdots & \ddots & \vdots \\ \sum_{i=1}^{p} \dfrac{\partial x_i}{\partial z_n} \cdot \dfrac{\partial y_1}{\partial x_i} & \cdots & \sum_{i=1}^{p}\dfrac{\partial x_i}{\partial z_n} \cdot \dfrac{\partial y_m}{\partial x_i}\end{array}\right]\\
&= \left[\begin{array}{ccc}\dfrac{\partial x_1}{\partial z_1} & \cdots & \dfrac{\partial x_p}{\partial z_1} \\ \vdots & \ddots & \vdots \\ \dfrac{\partial x_1}{\partial z_n} & \cdots & \dfrac{\partial x_p}{\partial z_n}\end{array}\right] \left[\begin{array}{ccc}\dfrac{\partial y_1}{\partial x_1} & \cdots & \dfrac{\partial y_m}{\partial x_1} \\ \vdots & \ddots & \vdots \\ \dfrac{\partial y_1}{\partial x_p} & \cdots & \dfrac{\partial y_m}{\partial x_p}\end{array}\right]\\
&=   \dfrac{\partial  X}{\partial Z} \dfrac{\partial  Y}{\partial X}
\end{align*}

Yes, the ordering matters as the dimensions won't allow multiplication if placed in the other way.

\end{solution}


\begin{problem}
Let $z:\R^p 
  \to \R$ be a function that depends on $\vx\in \R^p$ and let $Y$ be a $n$-vector that depends on $\vx\in \R^p$. Prove that
   $$
 \dfrac{\partial  }{\partial \vx} (z Y)=z
  \dfrac{\partial  Y}{\partial \vx}+
   \dfrac{\partial z}{\partial \vx} Y^T
 $$
   
\end{problem}
\begin{solution}
Given, $\vx \in \R^p$, $z(\vx) \in \R$ and $Y(\vx) \in \R^n$. Therefore,
\begin{align*}
\dfrac{\partial  }{\partial \vx} (z Y) &= \dfrac{\partial  }{\partial \vx} \left( \begin{bmatrix}z \cdot y_1\\\vdots\\ z \cdot y_n\end{bmatrix} \right)\\
&= \left[\begin{array}{ccc}\dfrac{\partial (z \cdot y_1)}{\partial x_1} & \cdots & \dfrac{\partial (z \cdot y_n)}{\partial x_1} \\ \vdots & \ddots & \vdots \\ \dfrac{\partial (z \cdot y_1)}{\partial x_p} & \cdots & \dfrac{\partial (z \cdot y_n)}{\partial x_p}\end{array}\right]\\
&= \left[\begin{array}{ccc}z \cdot \dfrac{\partial y_1}{\partial x_1} + y_1 \cdot \dfrac{\partial z}{\partial x_1} & \cdots & z \cdot \dfrac{\partial y_n}{\partial x_1} + y_n \cdot \dfrac{\partial z}{\partial x_1} \\ \vdots & \ddots & \vdots \\ z \cdot \dfrac{\partial y_1}{\partial x_p} + y_1 \cdot \dfrac{\partial z}{\partial x_p} & \cdots & z \cdot \dfrac{\partial y_n}{\partial x_p} + y_n \cdot \dfrac{\partial z}{\partial x_p}\end{array}\right]\\
&= z \dfrac{\partial Y}{\partial \vx} + \begin{bmatrix}\dfrac{\partial z}{\partial x_1}\\\vdots\\\dfrac{\partial z}{\partial x_p}\end{bmatrix} \begin{bmatrix}y_1 \cdots y_n\end{bmatrix}\\
&= z \dfrac{\partial Y}{\partial \vx} + \dfrac{\partial z}{\partial \vx} Y^T
\end{align*}

\end{solution}


\end{document}
