\documentclass[11pt]{amsart}
\usepackage{txfonts}
\usepackage{amsmath,amsthm,verbatim}
\usepackage{amscd,amssymb}
\usepackage[all]{xy} 
\usepackage{graphicx,calrsfs} 
\usepackage[colorlinks,plainpages,urlcolor=blue]{hyperref} 
\usepackage{breakurl}
\def\UrlBreaks{\do\/\do-}\usepackage{bbm}
\usepackage{bbm}
\usepackage{mathrsfs}
\usepackage{xstring}
\usepackage{tikz}
\usetikzlibrary{matrix,arrows,decorations.pathmorphing,calc,shapes}
\usepackage{mathtools}
\usepackage{tcolorbox}
\usepackage{wrapfig} 

\setlength{\parskip}{0.1 in}
\setlength{\parindent}{0pt}

%\usepackage[utf8]{inputenc}

\usepackage{listings}
\usepackage{xcolor}

\definecolor{codegreen}{rgb}{0,0.6,0}
\definecolor{codegray}{rgb}{0.5,0.5,0.5}
\definecolor{codepurple}{rgb}{0.58,0,0.82}
\definecolor{backcolour}{rgb}{0.95,0.95,0.92}

\lstdefinestyle{mystyle}{
   % backgroundcolor=\color{backcolour},   
    commentstyle=\color{codegreen},
    keywordstyle=\color{magenta},
    numberstyle=\tiny\color{codegray},
    stringstyle=\color{codepurple},
    basicstyle=\ttfamily\footnotesize,
    breakatwhitespace=false,         
    breaklines=true,                 
    captionpos=b,                    
    keepspaces=true,                 
    numbers=left,                    
    numbersep=5pt,                  
    showspaces=false,                
    showstringspaces=false,
    showtabs=false,                  
    tabsize=2
}

 \lstset{style=mystyle}



\topmargin=0in
\textwidth6.6in
\textheight8.6in
\oddsidemargin=-0.1in
\evensidemargin=-0.1in

\newtheorem{theorem}{Theorem} 
\newtheorem{corollary}[theorem]{Corollary}
\newtheorem{lemma}[theorem]{Lemma}
\newtheorem{prop}[theorem]{Proposition}

\theoremstyle{definition}
\newtheorem{definition}[theorem]{Definition}
%\newtheorem{example}[theorem]{Example}
\newtheorem{remark}[theorem]{Remark}
\newtheorem{conjecture}[theorem]{Conjecture}
\newtheorem{question}[theorem]{Question}
\newtheorem{problem}[theorem]{Problem}
\newtheorem{ex}[theorem]{Example}

 
\newcommand{\N}{\mathbb{N}}
\newcommand{\R}{\mathbb{R}}
\newcommand{\Q}{\mathbb{Q}}
\newcommand{\C}{\mathbb{C}}
\newcommand{\Z}{\mathbb{Z}}
\newcommand{\F}{\mathbb{F}}
\newcommand{\kk}{\mathbbm{k}}
\newcommand{\mm}{\mathbbm{m}}

\newcommand{\vtheta}{\vec{\theta}}


\newcommand{\va}{\vec{a}}
\newcommand{\vb}{\vec{b}}
\newcommand{\vc}{\vec{c}}
\newcommand{\vd}{\vec{d}}
\newcommand{\vh}{\vec{h}}
\newcommand{\e}{\vec{e}}
\newcommand{\vu}{\vec{u}}
\newcommand{\vv}{\vec{v}}
\newcommand{\vw}{\vec{w}}
\newcommand{\vx}{\vec{x}}
\newcommand{\vy}{\vec{y}}
\newcommand{\vz}{\vec{z}}
\newcommand{\Span}{\mathrm{Span}}
\newcommand{\red}{\textcolor{red}}
\newcommand{\blue}{\textcolor{blue}}
\newcommand{\green}{\textcolor{green}}
\newcommand{\ifif}{\textcolor{blue}{ if and only if }} 


\newcommand{\cB}{\mathcal{B}}
\newcommand{\cE}{\mathcal{E}}
\newcommand{\cC}{\mathcal{C}}
\newcommand{\cD}{\mathcal{D}}

\DeclareMathOperator{\im}{im}
\DeclareMathOperator{\codim}{codim}
\DeclareMathOperator{\coker}{coker}
 \DeclareMathOperator{\rank}{rank}
\DeclareMathOperator{\Nul}{\mathrm{Nul}}
\DeclareMathOperator{\Co}{\mathrm{Col}}
\DeclareMathOperator{\Ro}{\mathrm{Row}}
\DeclareMathOperator{\Ker}{\mathrm{Ker}}
\DeclareMathOperator{\Ima}{\mathrm{Im}}
\DeclareMathOperator{\dist}{\mathrm{dist}}
\DeclareMathOperator{\proj}{\mathrm{proj}}
\DeclareMathOperator{\refl}{\mathrm{ref}}
\DeclareMathOperator{\sign}{\mathrm{sign}}

\DeclareMathOperator{\id}{id}
\DeclareMathOperator{\tr}{tr}

\newcommand{\ef}{\textbf{ref}}
\newcommand{\rref}{\textbf{rref}}

\newcommand{\pdf}{\textbf{pdf }}
\newcommand{\pmf}{\textbf{pmf }}
\newcommand{\cdf}{\textbf{cdf }}

\newcommand{\normalcdf}{\textbf{normalcdf}}
\newcommand{\invNorm}{\textbf{invNorm}}
\newcommand{\invT}{\textbf{invT}}
\newcommand{\popdf}{\textbf{poissonpdf}}
\newcommand{\pocdf}{\textbf{poissoncdf}}

\DeclareMathOperator{\Var}{\mathrm{Var}}
\DeclareMathOperator{\Cov}{\mathrm{Cov}}
\DeclareMathOperator{\Corr}{\mathrm{Corr}}

\DeclareMathOperator{\Normal}{\mathrm{Normal}}

 \pgfmathdeclarefunction{gauss}{2}{%
  \pgfmathparse{1/(#2*sqrt(2*pi))*exp(-((x-#1)^2)/(2*#2^2))}%
}


  
 

\begin{document}
 \noindent
 
 %%%Black color
 
 Math 7243 
 
 \begin{center}
  \textbf{Homework 1.  Matrix calculus: } 
 \end{center}

Using
the denominator layout notation conventions.
\ \\
(One point each question except 4, 7,8 with 2 points each.)\\
Problems 4,7,8 have longer calculations.
  
\textbf{Problem 1. } Assume $\vx\in \R^n$,  $A\in \R^{m\times n}$ and $\vb\in \R^m$. Let $f(\vx)=\vb^T A \vx$. Find  $\nabla f$. 

\
 
 
\textbf{Problem 2. } Assume $\vx\in \R^n$. Find $\dfrac{\partial \vx^T\vx}{\partial \vx}$.
 
 \
 
 
\textbf{Problem 3. } Assume $\vx$ and $\va\in \R^n$. Find $\dfrac{\partial \big(\vx^T\va\big)^2}{\partial \vx}$
 
 
 \
 
\textbf{Problem $4$.} Suppose $\vx: \R^n\to \R^m$ is a map  sending $\vz\in \R^n$ to  $\vx(\vz)\in \R^m$. Similarly, 
suppose $\vy: \R^n\to \R^m$ and $A$ is an $m\times m$ constant matrix.
Prove that $\dfrac{\partial \big(\vy^TA\vx\big)}{\partial \vz}=
\dfrac{\partial \vy}{\partial \vz} A \vx+ \dfrac{\partial \vx}{\partial \vz} A^T \vy
$ 
 
\  
 
\textbf{Problem 5.}  Suppose $A(x): \R\to \R^{n\times n}$ is a map from $\R$ to $\R^{n
\times n}.$ 

Show that if $A(x)$ is invertible, then $
\dfrac{dA^{-1}}{dx}=-A^{-1} \dfrac{dA}{dx} A^{-1}
$
 
 
 \textbf{Problem 6. }
 Let $\vx$ and $\beta\in \R^p$. Prove that 
 $
\dfrac{\partial \vx^T\beta}{\partial \vx}=\beta 
  $
 
 
  \textbf{Problem 7. Chain Rule.}
 Assume that $Y$ is an $n$ vector but assume that $Y$ depends on $X$ and $X$ depends on
some $Z \in \R^q$. Show that
 $$
 \dfrac{\partial  Y}{\partial Z}= 
  \dfrac{\partial  X}{\partial Z}
   \dfrac{\partial  Y}{\partial X}
 $$
 Does the order matter? 
 
 Hint: This means that $X: \R^q\to \R^p$ and $Y: \R^p\to \R^n$.
 
  \textbf{Problem 8. } Let $z:\R^p 
  \to \R$ be a function that depends on $\vx\in \R^p$ and let $Y$ be a $n$-vector that depends on $\vx\in \R^p$. Prove that
   $$
 \dfrac{\partial  }{\partial \vx} (z Y)=z
  \dfrac{\partial  Y}{\partial \vx}+
   \dfrac{\partial z}{\partial \vx} Y^T
 $$
   
  
 
 
 
 


\end{document} 



 
